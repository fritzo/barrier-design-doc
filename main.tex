%%%% Small single column format
\documentclass[anonymous=false, %
               format=acmsmall, %
               review=true, %
               screen=true, %
               nonacm=true]{acmart}

\usepackage[ruled]{algorithm2e} 
%\usepackage{parskip}

\urlstyle{tt}
\citestyle{acmauthoryear}

\begin{document}

\title{Delayed Sampling with Funsors}
%  \titlenote{This is a titlenote}
%  \subtitle{This is a subtitle}
%  \subtitlenote{Subtitle note}

\author{Fritz Obermeyer}
%\orcid{1234-5678-9012-3456}
\affiliation{%
  \institution{Uber AI}
  %\department{}
  %\streetaddress{43 Vassar St}
  %\city{Cambridge}
  %\state{MA}
  %\postcode{02139}
  %\country{USA}
}
\email{fritzo@uber.com}

\author{Eli Bingham}
%\orcid{1234-5678-9012-3456}
\affiliation{%
  \institution{Uber AI}
  %\streetaddress{625 Mt Auburn St #3}
  %\city{Cambridge}
  %\state{MA}
  %\postcode{02138}
  %\country{USA}
}
\email{eli.bingham@uber.com}
%\renewcommand\shortauthors{Mage, M. et al}

\begin{abstract}
Delayed sampling is an inference technique for automatic Rao-Blackwellization in sequential latent variable models.
Funsors are a software abstraction generalizing Tensors and Distributions and supporting seminumerical computation including analytic integration.
We demonstrate how to easily implement delayed sampling in a Funsor-based probabilistic programming langugae using effect handlers for \texttt{sample} statements and a \texttt{barrier} statement.
\end{abstract}

\maketitle

\section{Introduction}

\section{Delayed Sampling}

Delayed sampling \cite{murray2017delayed} is an inference technique for automatic Rao-Blackwellization in sequential latent variable models.
Delayes sampling was introduced in the Birch probabilistic programming language \cite{murray2018automated}.

\section{Funsors}

Funsors \cite{obermeyer2019functional} are a software abstraction generalizing Tensors and Distributions and supporting seminumerical computation including analytic integration.

\section{Delayed Sampling with Funsors}

Consider a probabilistic programming language with two statements and two effect handlers:
\begin{itemize}
  \item The statement \verb$x = sample(name,dist)$ is a named stochastic statement, where \verb$x$ is a Funsor value, \verb$name$ is a unique identifier for the statement, and \verb$dist$ is a Funsor distribution.
  \item The statement \verb$x = barrier(x)$ elimiates any free variables from the recursive data structure \verb$x$, which may contain Funsor values.
  \item The effect handler \verb$condition(name, data)$ conditions a model to observed \verb$data$ andd affects only the single \verb$sample$ statement with matching \verb$name$.
  \item The effect handler \verb$log_joint()$ records a representation of the cumulative log joint density of a model as a Funsor expression.
\end{itemize}
We assume this probabilistic programming language is embedded in a host language, and use Python for the host langugage in this paper.

\bibliographystyle{acm-reference-format}
\bibliography{main}

\appendix

\end{document}
